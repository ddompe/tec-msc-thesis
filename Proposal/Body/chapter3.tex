\chapter{Methodology and Schedule} 
\section{Methodology}
Describe in technical language your research perspective and your past, present, or possible future points of view.
List three research methodologies you could use, and describe why each might be appropriate and feasible. Select the most viable method.

\section{Definition of Terms}
Describe for the reader the exact meaning of all terms used in the problem, purpose and methodology sections. Include any terms that, if not defined, might confuse the reader.
State the clearest definition of each term using synonyms, analogies, descriptions, examples etc. Define any theoretical terms as they are defined by proponents of the theory you are using.

\section{Assumptions}
Describe untested and un-testable positions, basic values, world views, or beliefs that are assumed in your study.
Your examination should extend to your methodological assumptions, such as the attitude you have toward different analytic approaches and datat-gathering methods. Make the reader aware of your own biases.

\section{Scope and Limitations}
Disclose any conceptual and methodological limitations
Use the following questions to identify the limitations of your study: What kind of design, sampling, measurement, and analysis would be used “in the best of all possible worlds”? How far from these ideals is your study likely to be?

\section{Procedure}
Describe in detail all the steps you will carry out to choose subjects, construct variables, develop hypotheses, gather and present data, such that another researcher could replicate your work.
Remember the presentation of data never speaks for itself, it must be interpreted.

\section{Long-Range Consequences}
Think ahead approximately three years after the completion of your thesis project. What are the long-term consequences of your having done the study or not done the study?
If you carry out the study successfully your results will: confirm your hypothesis; contradict your hypothesis; or possibly be inconclusive.

\section{Work plan}
