% Now let put the abstract in two languages
\doublespacing

\selectlanguage{english}%
\begin{abstract}
Embedded Systems are one of the fastest growing markets in the computer industry, but the discipline of embedded software development is still young; best practices common on other software development areas are not encouraged by the market dynamics, product development cycles nor the requirements of low-level integration with the hardware.  Furthermore, embedded developers tend to shy away from existing frameworks that may improve re-usability when they ca not meet their performance goals. This is particularly true for multimedia systems, where the amounts of data processing may easily impose severe overhead in an embedded architecture.
All these factors lead to low re-usability of the software written for embedded systems.

This work addresses the problem of improving multimedia software stacks for embedded systems, focusing on the GStreamer multimedia framework with the Linux operating system on the ARM architecture (specifically the DM36x and DM37x Systems On a Chip from Texas Instruments). The work will focus on characterizing the performance penalties and overhead induced by the framework and operating system in a set of specific multimedia scenarios, and propose and implement changes to the algorithms and data structures in order to improve performance and reduce penalties. It will also contribute with knowledge on the tools and methods required to understand the complex behavior of multimedia software stacks.
\end{abstract}

\selectlanguage{spanish}%
\begin{abstract}
Los sistemas empotrados son uno de los mercados con mayor crecimiento en la industria de la computación, sin embargo la disciplina de desarrollo de software empotrado es todavía joven; las mejores prácticas que son comunes en otras áreas de desarrollo de software no son propiciadas por las dinámicas del mercado, los ciclos de desarrollo de productos y el requerimiento de integración de bajo nivel con el hardware. Los desarrolladores de sistemas empotrados suelen evitar ambientes de desarrollo  que pueden mejorar la re-usabilidad si los mismos no cumplen con los objetivos de desempeño. Esto es particularmente cierto para sistemas de multimedios, donde las grandes cantidades de procesamiento de datos pueden fácilmente imponer sobrecarga en una arquitectura empotrada. Todos estos factores llevan a una baja re-usabilidad del software escrito para sistemas empotrados.

El presente trabajo se enfoca en el problema de mejorar los ambientes de trabajo de multimedios para sistemas empotrados, enfocándose en el ambiente GStreamer para el sistema operativo Linux en la arquitectura ARM (especificamente en los SoC DM36x y DM37x de Texas Instruments). El trabajo se enfocará en caracterizar las penalidades de desempeño y sobrecarga generados por el ambiente multimedia y el sistema operativo en un conjunto de escenarios específicos, y proponer e implementar cambios a los algoritmos o estructuras de datos para mejorar el desempeño y reducir las penalidades. Este trabajo también contribuirá con entendimiento de las herramientas y metodologías para comprender los comportamientos complejos de los ambientes de software multimedia.
\end{abstract}

\selectlanguage{english}%

\singlespace