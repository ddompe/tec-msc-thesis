\chapter{Introduction and background}
\epigraph{People in the embedded space don't do prototypes. They hack something until it works, then it's done.}{xav user on lwn.net}

\section{Introduction}
Embedded Systems are one of the fastest growing markets in the computer industry, with more than 10 billion embedded processors shipped in 2008 alone\citep{Clarke:2009uq}. We can define embedded systems as microprocessor-based systems built to control a function or range of function, not designed to be programmed by the end user in the same way that a \ac{PC} is\cite{heath2003embedded}.

Embedded Software development as discipline is younger than software development as intended for other environments like \acp{PC} or Mainframes. There are some particular issues that transform embedded software development into a complex eco-system:
\begin{itemize*}
\item It usually requires a significant level of integration with the hardware that is uncommon in other software development areas.
\item Most of the hardware integration that is done is very particular and unique, which usually makes the code hard to re-use if not designed properly (or if the technology is new and it isn't clear how to implement the software solution in a generic way).
\item The market for embedded system is so fast-paced that developers are constantly pressured to deliver working devices as soon as possible. Few companies care about the quality or re-usability of their embedded code, except for hardware vendors or efforts from single open-source minded developers who want to contribute back. For example recent articles regarding the status of the contributions to the Linux™ kernel for the \ac{ARM} architecture qualify these contributions' status as a  ``mess'' due to the many different interests and agendas behind it \cite{Proffitt:2011fk}.
\item Many embedded developers may lack formal training on software development, but instead have an \ac{EE} background.
\end{itemize*}

These factors usually mean that embedded developers will not likely find or apply the same best-practices that the rest of the software industry uses. It is not rare to find embedded projects where each revision of the same product will have a completely new software stack or re-implementation from the previous iteration.

The fact that the code is tailored for specific hardware to the point where it can't be easily re-used goes against \acp{OS} theory which proposes that the \ac{OS} should abstract all the software from the hardware implementation \citep[p.~29]{Silberschatz:2010vn}. However many embedded devices (especially ones with low memory footprint) may use a custom-made \ac{OS} or use no \ac{OS} at all (also called \ac{OS}-less solutions).

\subsection{Linux on embedded systems: relevance and challenges}
On embedded systems capable enough to support it, Linux has gradually gained acceptance as an \ac{OS} due to its open-source nature and support from embedded processors manufacturers, especially on the \ac{SoC} market. \acp{SoC} are the main trend for new product development in recent years due to the advantages provided by new chip development \citep{Somaya:2000fk}.

The advantages that Linux provides for the development of new embedded products  are clear:
\begin{itemize*}
\item It's a widely deployed \ac{OS} which has been successfully used in different embedded markets.
\item No royalty fees or licenses costs are associated directly with it (although some embedded vendors may add royalty fees if you use their custom distributions).
\item Wide knowledge base available and many developers are familiarized with it.
\item Usually hardware companies provide drivers and support for Linux in their latest devices (for example ethernet or wireless circuits, storage devices, etc.), simplifying the integration of hardware components for an embedded system. This is a critical element for minimizing the time-to-market of a product.
\end{itemize*}

However Linux also presents some challenges for adaption within the embedded space:
\begin{itemize*}
\item It's not designed to be a \ac{RTOS}, which make it not suitable for all type of embedded applications.
\item There is a learning curve for \ac{RTOS} developers, since Linux enforces some concepts that may be foreign to them:
	\begin{itemize*}
	\item Memory space protection: most \ac{RTOS} do not provide memory space protection.
	\item Separation between the kernel and user space applications: This is a side-effect of the memory space protection.
	\end{itemize*}
\item There are several algorithms used in non-\acp{RTOS} like Linux that may affect the performance compared to \acp{RTOS} or \ac{OS}-less solutions:
	\begin{itemize*}
	\item Context Switches: there is a penalty associated on several hardware architectures with the memory protection.
	\item I/O Scheduling: the kernel page cache may defer I/O work. Linux offers full control to customize this behavior for embedded systems, but many developers may be unaware of it.
	\item Interrupts prioritization: mainline Linux kernel doesn't offer a way to control priorities for interrupt handlers. There are patches to transform interrupt handlers into kernel threads, but they aren't available for all kernel versions and may require specific analysis and tuning to achieve the desired results \cite{Song_Chubb_10}.
	\end{itemize*}
\end{itemize*}

Despite the short-comings that Linux presents versus a \ac{RTOS}, just the fact of having widely available drivers and software stacks for peripherals on the embedded space usually outweighs these problems. Many \acp{SoC} may require complex software stacks to interact with integrated peripherals; such is the case of heterogenous \acp{SoC} using a \ac{GPP} and a \ac{DSP} or other sort of hardware unit for accelerating specific operations (such as video coding, cryptographic algorithms, etc). Implementing or porting the functionality of these software stacks required to interact with the desired peripheral to an \ac{OS}-less or other non-supported \ac{RTOS} is usually non-effective in terms of cost or time.

However many developers that are un-familiar with Linux and/or general purpose \acp{OS} design, approach the task of implementing the functionality for their embedded products bypassing Linux functionality. For example the author of this work have found several times kernel drivers for Linux designed just to specifically allow user-space access to the peripherals by any user space application in commercial \ac{OEM}/\ac{ODM} offerings for embedded development.  There are several reasons for developers using this kind of approach:
\begin{itemize*}
\item Some times the software stack was designed in \ac{OS}-less environment and ported over to Linux. Developers just try to find a way to get the Linux hardware management out of the way of the software stack.
\item Developers are simple un-familiar with Linux design.
\item Linux may lack APIs or functionality to handle new scenarios posted by the hardware design. For example ability to realize zero-memory-copy operations.
\item Developers try to use Linux's APIs but found that performance was un-acceptable for unknown reasons. In the embedded community is widely believed that general purpose \ac{OS} like Linux may be unsuitable for certain embedded applications since is considered ``bloated'' for desktop systems.
\end{itemize*}

Another common believe in the embedded community is that any software design highly specialized will provide better performance than a implementing it over a generic solution. One example of this is multimedia software architectures on top of Linux, where the amount of data being processed requires to use data paths with zero memory copy operations across the whole software stack.
 
\section{Problem statement}
Many embedded developers tend to create software designs and architectures that aren't re-usable or maintainable for different reasons; one of them is the empiric premise that generic software architectures penalize performance. Furthermore there are little incentives from the product development cycle for embedded products to invest into optimizing the \acp{OS}. All these factors lead to creation of software platforms that are hard to reuse or leverage into future project versions or other projects, potentially increasing the costs of development.

The described scenario is particular true for multimedia software stacks for embedded systems. Multimedia software stacks moves big amounts of data at very high speeds between several hardware devices and across different subsystems in the \ac{OS}. This work focus on improving the performance of GStreamer, a multimedia software stack for Linux, on the \ac{ARM} architecture. GStreamer is a popular solution for embedded systems, and there are known issues with his performance but not complete clarity about the source of the them\cite{Contreras:vn}.

\section{Hypothesis}
It's possible to improve multimedia software stacks in general purpose \aclp{OS} like Linux to minimize processing overhead to be negligible in terms of CPU, memory and I/O requirements.

\section{Previous work}
Plagemann et al.\cite{Plagemann2000267} provided a very comprehensive review of the existing work in the field regarding customization of \acp{OS} algorithms for multimedia scenarios respect to CPU and disk scheduling, resource allocation, filesystem structures, memory management, I/O subsystems. Their work resumes different approaches and developments from the 90's decade, but the publication is now 12 years old and it doesn't seems to be more up-to-date surveys as comprehensive as this work.

\ac{LMWG} has begun to work recently on similar problems\cite{Reis:vn} to the stated on this work. One of the short terms objectives of \ac{LMWG} is to work on multimedia validation tools to improve the understanding of system behavior. The scope of the present work will be narrower than the aimed by \ac{LMWG}, and eventually this work could be a contribution to it.

\subsection{Literature review}
Performance issues can be analyzed from the perspective of the system design. There are classic papers that address the different design approaches for general purpose \aclp{OS} and their respective tradeoffs \cite{Engler95exokernel:an}\cite{Bershad95extensibility}\citep{Liedtke:1995kx}. Embedded \acp{RTOS} are usually designed without memory or process protection; an approach that suits well for direct-access to the hardware from the applications, eliminates any type of overhead induced by the \ac{OS} and allows deterministic latency in the scheduler.

The debate regarding general purpose \aclp{OS} design is a long standing and recurring topic in Computer Science (the Tanenbaum-Torvalds debate \cite{DiBona:1999:OSV:553109} have came up recently again\cite{Tanenbaum:fk} and even in other forms by different proponents\cite{Heiser:uq}). 

Of particular interest is Liedtke\citep{Liedtke:1995kx} work, since he showed that using the right algorithms and analysis it's possible to implement software stacks that manage hardware access with minimal performance penalties. He faced common belief that $\mu$-kernels suffer inherit design performance penalties and demonstrated that rather than a design issue it was an implementation issue the root cause of the performance problems. The problem of multimedia stack performance on embedded systems could be viewed as a generalization of the same problems faced by Liedtke:
\begin{itemize}
\item Common belief that the existing implementation is inherently inefficient by design.
\item The problem can be approached by improving the algorithms and data structures used in the existing implementation.
\end{itemize}

However a fundamental part to approach the issue of performance improvements is to find tools for validate the different dynamic scenarios. There is previous published work by this author regarding instrumenting GStreamer for specific multimedia scenarios\cite{Dompe:2011ys} that includes some tools and methodologies to analyze the performance and, as mentioned previously, developing these tools is also an objective of the \ac{LMWG} too. Some simple tests has been written by GStreamer developers when analyzing specific performance issues\cite{Contreras:vn} as well, however these tools are usually limited to instrumenting GStreamer and not the complete system behavior or characterization of his interactions with the different Linux subsystems.

To properly address the different factors involved on the performance of multimedia software stacks for embedded systems the work has to be split up in different areas:
\begin{itemize}
\item CPU scheduling and Memory Protection: scheduling and real-time behavior taking in consideration the tradeoffs imposed by the memory protection. On modern \acp{SoC} the scheduling should consider offloading work to co-processors in boss-worker modes.
\item I/O utilization: data structures and algorithms for properly handling of I/O from and among different subsystems on modern complex platforms like heterogenous multi-core \acp{SoC}.
\end{itemize}

The following section address existing literature regarding multimedia software stacks and the different areas mentioned previously.

\subsection{CPU Scheduling and Memory Protection}
Poudel\cite{Poudel:ys} reported notable performance gains in data throughput, during development of the OpenCV DSP Acceleration project with the use of asynchronous processing APIs to the coprocessor (\ac{DSP}). Even if this is not an scheduling algorithm per-se it shows the relevance of accounting for the nature of the processing architecture in the final throughput of the system. Regehr et al.\cite{Regehr00operatingsystem} explore the relevance of the scheduling algorithms and the programming model to properly take advantage in the design of multimedia applications.

One of the main performance penalties and predictability for general-purpose \acp{OS} are due the memory protection semantics\cite{Blackham_SH_11}. Liedtke\citep{Liedtke:1995kx}  work showed that is important to provide architecture specific optimizations, so we need to explore work regarding optimizations for the \ac{ARM} architecture. David et al.\citep{David:2007:CSO:1281700.1281703}  found significant context switch overhead for Linux in the \ac{ARM} platforms. Chanteperdrix and Cochran \citep{Chanteperdrix:2009fk}) among others have implemented solutions to address this problems with different limitations, but have never made it to the mainline Linux kernel.

\subsection{I/O utilization}
There is many previous work regarding optimizing \acp{OS} for efficient I/O handling of multimedia cases, with special focus on optimized network streaming from disk (YIMA\cite{Shahabi:2002vn},MARS\cite{Jane98enhancementsto}), or optimizing the networking stack for zero-memory-copy (LyraNet\cite{10.1109/RTCSA.2005.57},  EIORTP\cite{springerlink:10.1007/s11390-006-0989-5}, Solaris\cite{Chu96zero-copytcp}). However these works doesn't address modern hardware requirements of embedded \ac{SoC}s like other I/o subsystems beyond networking.

Work addressing improved I/O throughput on modern \ac{SoC} architectures is found in some conference presentations for the SH processor\cite{Parker:2009fk}, and the OMAP4\cite{Clark:2010uq}. These work present useful information regarding community work to address some of the problems that will be researched by this thesis.

\section{Long-Range Consequences}
This work will contribute to the analysis of the factors involved on measuring and improving performance the Linux kernel for his use on embedded systems dealing with multimedia software stacks. Such contributions would contribute to industry-wide efforts like the \ac{LMWG}. 
