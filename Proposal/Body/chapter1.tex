\chapter{Introduction and background}
%\epigraph{⟨text⟩}{⟨source⟩}

\section{Introduction}
Should be 1-2 pages
write it so it captures the reader's interest in this overview. It does not have to be perfect.
You can write this section last. Your best overview of you project most likely will come after you have written the other sections of your proposal.

\section{Background}
Key ideas

Embedded Systems are everywere

Embedded Systems change quickly and evolve faster than previous technologies, thanks to technologies like \ac{SoC}, which revolutionize the way the technology industry works\cite{Somaya:2000fk}. For example on \acp{DSP}

Tradeoff between custom hardware vrs software. \acp{DSP} are good solution (why?). Advantage of heterogenous mixed environments on the operating system level.

Software complexity for DSP programming on heterogenous architecture. Hardware complexity interfering with the software.

Software re-use for the ARM side with a framework like gstreamer.

Get into details for the target platforms. Mention at least the platforms where the accelerators are not general propose \acp{DSP}.

\section{Problem statement}
Integrating custom DSP algorithms on higher-level frameworks in order to gain time-to-market is a difficult tasks and requires detailed knowledge of the intrinsics of the platform to get it right. Not all the knowledge about the challenges of integration are documented or measured.

\section{Previous work}

\subsection{Literature review}
Locate and briefly describe those studies and theories that support and oppose your approach to the problem. In other words, place the proposed study in context through a critical analysis of selected research reports.
Be sure to include alternative methodological approaches that have been used by others who studied your problem.

\subsection{Previous solutions}
Describe previous attempts and similar solutions:
- gst-ti plugin
- gst-ti 2.x plugin
- gst-openmax
- gst-ducati

\section{Purpose}
The purpose of this study is to create a software architecture to integrate existing and new DSP algorithms into the higher level multimedia framework. This framework should be efficient from the perspective of hardware constrains and leverage existing knowledge from previous experiences.

\section{Long-Range Consequences}
Will improve time to market for products based on the platforms supported by the target framework. It will provide reference code for optimized software architecture for resource-constrained systems.
