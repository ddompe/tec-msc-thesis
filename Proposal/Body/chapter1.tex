\chapter{Introduction and background}
%\epigraph{⟨text⟩}{⟨source⟩}

\section{Introduction}
Should be 1-2 pages
write it so it captures the reader's interest in this overview. It does not have to be perfect.
You can write this section last. Your best overview of you project most likely will come after you have written the other sections of your proposal.

\section{Background}
Capture the reader's interest and convince him/her of the significance of the problem.
Give at least three reasons why the problem you have chosen is important to you and society, and specify at least two concrete examples of the problem.
\nomenclature{DSP}{Digital Signal Processor}

\section{Problem statement}
First formulate a research question. Next restate the question in the form of a statement: note the adverse consequences of the problem.
The type of study determines the kinds of question you should formulate, such as Is there something wrong in society, theoretically unclear or in dispute, or historically worth studying? Is there a program, drug, project, or product that needs evaluation? What do you intend to create or produce and how will it be of value to you and society?

\section{Previous work}

\subsection{Literature review}
Locate and briefly describe those studies and theories that support and oppose your approach to the problem. In other words, place the proposed study in context through a critical analysis of selected research reports.
Be sure to include alternative methodological approaches that have been used by others who studied your problem.

\subsection{Previous solutions}

\section{Purpose}
Begin with “The purpose of this study is to…” change, interpret, understand, evaluate, or analyze the problem.
State your thesis goal completely. Remember, it should be some form of investigative activity.\

\section{Long-Range Consequences}
Think ahead approximately three years after the completion of your thesis project. What are the long-term consequences of your having done the study or not done the study?
If you carry out the study successfully your results will: confirm your hypothesis; contradict your hypothesis; or possibly be inconclusive.
