\chapter{Objectives and Contributions}
\epigraph{Don't find fault, find a remedy.}{Henry Ford}

\section{General Objective}

\begin{itemize}
\item Measure and analyze the possible performance bottlenecks in the design of a general-propose \acl{OS} (specifically in Linux) in an embedded system for handling of multimedia data streams to propose changes in the algorithms and data structures to improve the performance for such data scenarios.
\end{itemize}

\section{Specific Objectives}
\begin{itemize*}
\item To instrument and measure performance of the Linux kernel in an specific set of embedded systems specialized for multimedia data processing, using an specific set of multimedia processing scenarios.
\item To design and execute experiments to identify the sources of performance bottlenecks in the selected scenarios.
\item Document and characterize the results from the experiments for performance measurement.
\item Propose algorithms or changes required to improve performance.
\end{itemize*}

\section{Contributions to the subject}
This work will provide documentation regarding the details of the specific tradeoffs for the implementation approaches of multimedia software on the target platforms, and could serve as basics for future work regarding improving performance (for example exploring improvements for power consumption reduction on the specific scenarios).

\section{Scope and Limitations}
\begin{itemize*}
\item Work will be done specifically using the multimedia framework GStreamer on Linux embedded systems running on \acp{SoC} from Texas Instruments on two different architectures:
	\begin{itemize*}
	\item OMAP3x architecture: this \ac{SoC} has a \ac{DSP} co-processor to offload codec processing.
	\item DM36x architecture: this \ac{SoC} has non-programable hardware accelerators to offload codec processing.
	\end{itemize*}
\item Implementations of the proposed algorithms and improvements will be made, but no guarantee that such changes will be merged back into mainstream open source projects is done. Still the work will be done interacting with the respective project maintainers with the idea of creating changes that can be merge on mainstream.
\end{itemize*}
 