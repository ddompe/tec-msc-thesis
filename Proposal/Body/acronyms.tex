\chapter{Acronyms}

% Please keep it sorted by alphabetical order
\begin{acronym}

\acro{ARM}{Advanced RISC Machine} is a hardware architecture designed by ARM Holdings and licensed to many companies worldwide.

\acro{DSP}{Digital Signal Processor} is a special type of micro processor.

\acro{EE}{Electronic Engineering} is the engineering area dedicated to electronic circuits technologies.

\acro{ECC}{Error Correction Codes} are data used to calculate and correct errors in blocks of data.

\acro{FTL}{Flash Transition Layer} is a software stack that allows to access flash storage with the same semantics for block read/write as standard hard disk, taking care of wear-leveling for the flash device and \ac{ECC} handling on flash that require it.

\acro{GPP}{General Purpose Processor} is a processor suited for standard computational operations, in contrast of a \ac{DSP} for example.

\acro{LMWG}{Linaro's Multimedia Working Group} is working group inside Linaro project that focus on Multimedia improvements for Linux on the \ac{ARM} platform.

\acro{OEM}{Original Equipment Manufacturing} refers to a company or a firm that is responsible for designing and building a product according to its own specifications, and then selling the product to another company or firm, which is responsible for its distribution. The one company produces products on behalf of another company, after which the purchasing company markets the product under its own brand name\cite{:uq}.

\acro{ODM}{Original Design Manufacturing} is a company or firm that is responsible for designing and building a product as per another company’s specifications\cite{:uq}.

\acro{OS}{Operating System}

\acro{PC}{Personal Computer}

\acro{RTOS}{Real Time Operating System} is an operating system that is designed to meet with Real Time deadlines.

\acro{SoC}{System on a Chip} is a technology that consist on integrating several functional blocks of re-usable electronics logic (even from different vendors) into single die.
\acrodefplural{SoC}[SoCs]{Systems on a Chip}

\end{acronym}

